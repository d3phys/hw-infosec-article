\documentclass[a4paper,12pt]{article}


\usepackage[english,russian]{babel}	% локализация и переносы
%\usepackage{minipage}

\usepackage{unicode-math}
\usepackage{fontspec}
\setmainfont{Garamond}
\setmathfont{Garamond-Math}

%%% Гиперссылки
\usepackage{hyperref}
\usepackage[usenames,dvipsnames,svgnames,table,rgb]{xcolor}
\hypersetup{				% Гиперссылки
	unicode=true,           % русские буквы в разделареальных установках
	pdfproducer={Производитель}, % Производитель
	pdfkeywords={keyword1} {key2} {key3}, % Ключевые слова
	colorlinks=true,       	% false: ссылки в рамках; true: цветные ссылки
	linkcolor=red,          % внутренние ссылки
	citecolor=green,        % на библиографию
	filecolor=magenta,      % на файлы
	urlcolor=cyan           % на URL
}

\usepackage{geometry} % Простой способ задавать поля
\geometry{top=26mm}
\geometry{bottom=26mm}
\geometry{left=26mm}
\geometry{right=26mm}

\setcounter{secnumdepth}{2}

\usepackage{tikz}
\usepackage{graphicx}


\begin{document}
	
\begin{center}
	
        \normalsize Московский физико-технический институт \\
						(государственный университет)
	\vspace{46ex}
	

	\Large \textbf{Аппаратные решения для задач \\ защиты информации}

	\vspace{40ex}
\end{center}
\begin{flushright}
	\normalsize{Автор: Дедков Денис Андреевич \\ студент Б01-108 \\}
\end{flushright}

\vfill

\begin{center}
	Долгопрудный, 2024
\end{center}

\thispagestyle{empty} % выключаем отображение номера для этой страницы

\newpage

\tableofcontents{}
\newpage

\section{Введение}

Акцент поставлен на применение в развивающихся, современных областях. Выделены нерешенные проблемы создания таких решений.

Обзор существующих аппаратных решений для актуальных задач в сфере защиты информации.

Приведены количественные  со ссылками на соответствующие исследования и научные работы. 

Проводится классификация аппаратных решений. систематизация существующей информации.

Ссылки на передовые решения 


\section{Причины использования аппаратных решений}

Когда речь идет об аппаратных решениях, чаще всего имеют в виду ASIC (Application-Specific Integrated Circuit, интегральная схема специального назначения). Такие микросхемы выполняют строго ограниченные функции. Вследствие этого выполнение функций происходит более эффективно. Следствием специализации также является конечная стоимость производства таких чипов: она может быть на порядки меньше стоимости чипа общего назначения. 

Любому, будь то SoC (System on Chip, Система на Кристалле) или ASIC, решению соответствует точка в пространстве PPA (Power, Performamce, Area). \footnote{Энергопотребление, производительность и площадь чипа.} По расположению чипа в этом пространстве, можно определить его специализацию, или мощность множества задач, который данный чип эффективно может решать при реалистичной стоимости производства (см. рис. \ref{fig:asic-space}). Именно с этой точки зрения ASIC решения являются наиболее выгодными.

\begin{figure}[h]
	\centering
	\begin{tikzpicture}
		\draw[->] (0, 0) -- (5, 0) node[right] {Производительность};
		\draw[->] (0, 0) -- (0, 5) node[above] {Энергоэффективность};
		\draw[dashed] (0, 0) -- (4, 4) node[right] {Специализация vs. Гибкость};
		\draw[<->] (0.5, 1.5) -- (1.8, 0.2) {};
		\draw[<->] (1, 1) -- (3, 3) ;
		
		\draw[dashed] (0.5, 0) -- (0.5, 4.5) node[right] {Производит. ограничения};
		\draw[dashed] (0, 0) -- (4.5, 0.5) node[right] {Энергоэфф. ограничения};
		
		\node[label=above:{ASIC}] (quantum) at (3,1.75) {};
		\draw[fill] (2,2) circle (0.05cm) node (ASIC_p) {} ;
		
		\draw[fill] (1.8,0.2) circle (0.05cm) node (HighPerf_p) {} ;
		\draw[fill] (0.5,1.5) circle (0.05cm) node (Embed_p) {} ;
		
		\draw[fill] (1,1) circle (0.05cm) node (CPU_p) {} ;
		\node[label=above:{CPU}] (CPU) at (1.6,0.5) {};
		
		\node[label=above:{Высокопроизв. решения}] (HighPerf) at (-5,1) {};
		\node[label=above:{Энергоэфф. решения}] (Embed) at (-3,3) {};
		
		\draw[->] (HighPerf) to [bend right=40] (HighPerf_p) ;
		\draw[->] (Embed) to [bend right=26] (Embed_p) ;
	\end{tikzpicture}
	
	\caption{Пространство вычислителей. Рисунок на основе курса \cite{DigitalASICDesign}.}
	\label{fig:asic-space}
\end{figure}

Не удивителен тот факт, что специализированные аппаратные решения востребованы и в сфере защиты информации, включающей в себя огромные массивы вычислительных задач. А применение таких чипов продолжает набирать популярность. 

Предлагается разделить существующие аппаратные решения на два пересекающихся класса: \textbf{ускорители} и \textbf{усилители}.

Основное применение ускорителей - улучшение производительности систем защиты информации. Чаще всего под этим подразумевается ускорение работы существующих крипторафических алгоритмов. Чистые ускорители не привносят новые механизмы защиты и, с точки зрения функциональности, могут быть заменены решениями общего назначения.

Основное применение усилителей - улучшение безопасности систем защиты информации. Такое усиление может достигаться как путем изоляции криптографических вычислений, так и исключительно новыми подходами в обеспечении защиты, вроде применения физических принципов. Такие решения могут работать существенно медленней систем на основе чипов общего назначения, однако их использование приводит к колоссальному росту защищенности, что и является основной причиной их использования.

На рисунке \ref{asic-classes} отображены классы решений и некоторые их реализации, обзор которых и является одной из основных целей данной работы.  



\begin{figure}
% Definition of circles
\def\firstcircle{(0,0) circle (2.5cm)}
\def\secondcircle{(0:4cm) circle (2.5cm)}
\def\secondcircle{(0:4cm) circle (2.5cm)}

\colorlet{circle edge}{blue!50}
\colorlet{circle area}{blue!20}

\tikzset{filled/.style={fill=circle area, draw=circle edge, thick},
	outline/.style={draw=circle edge, thick},
	mycirc/.style={circle,fill=black!20, minimum size=0.01cm}}

\setlength{\parskip}{5mm}

\begin{center}
	\begin{tikzpicture}
		
		\node[label=above:{GPGPU}] (gpgpu) at (-1cm,3.6cm) {} ;
		\node[label=above:{FPGA}] (fpga) at (3cm,3.5cm) {} ;
		\node[label=above:{CPU Extensions}] (cpu) at (2cm,4.5cm) {} ;
		\node[label=above:{Квантовые решения}] (quantum) at (7cm,3cm) {};
		
		\draw (0,0) circle (3cm) node (accelerators) {Ускорители} ;
		\draw (4.5cm,0) circle (3cm) node (opportunitie) {Усилители} ;
		
		m	\draw (-1.2cm,1cm) circle (0.1cm) node (gpgpu_p) {} ;
		\draw (2.5cm,1cm) circle (0.1cm) node (fpga_p) {} ;
		\draw (0.5cm,1.5cm) circle (0.1cm) node (cpu_p) {} ;
		\draw (6cm,1.5cm) circle (0.1cm) node (quantum_p) {} ;
		
		\draw[->] (gpgpu) to [bend right=26] (gpgpu_p) ;	
		\draw[->] (fpga)  to [bend left=20] (fpga_p) ;	
		\draw[->] (cpu)  to [bend left=10] (cpu_p) ;	
		\draw[->] (quantum)  to [bend right=25] (quantum_p) ;	
	\end{tikzpicture}
\end{center}
\caption{Классы аппаратных решений.}
\label{asic-classes}
\end{figure}

\section{Обзор существующих аппаратных решений}

\subsection{Расширения CPU}

Производители центральных процессоров общего назначения (General-purpose CPU) интегрируют в чип специализированные блоки, позволяющие ускорять вычисления. Такие микроархитектурные изменения отображаются в архитектуре расширением системы команд.

Есть две основных причины успеха такого рода расширений: использование широких регистров (длиной 128 бит и более) и внедрение специальных вычислительных блоков. 

Векторные регистры позволяют производить вычисления над пачкой из нескольких 8, 16, 32 или 64-битных чисел одновременно, улучшая тем самым пропускную способность (bandwidth) чипа.

Специализация позволяет заменить последовательность из десятка операций центрального процессора общего назначения на одну-две инструкции, исполняемые на отдельном вычислительном блоке, чем уменьшает общее время, требуемое на совершение операции (latency).  

Два десятилетия назад началось внедрение расширений CPU \footnote{Речь идет о MMX расширении процессоров компании Intel.}, что привело к удачной коллекции векторных расширений: MMX, SSE, AVX архитектур Intel 64 и AMD64, Neon расширение архитектуры Arm, расширение V системы команд RISC-V. Векторные расширения, позволяя ускорять вычисления в десятки раз, завоевали большую популярность, чем положили начало массовому росту количества и разнообразия таких расширений.

Как следствие был создан целый набор расширений для задач защиты информации: Intel SHA (SHA-1, SHA-256, SHA-512), AES. К примеру, Arm v8 Crypto Extensions показывают ускорение SHA2-256 до $6x$ относительно оригинального алгоритма на процессоре общего назначения, в зависимости от размера блока \cite{AMD_VASoC}.

Недавно актуальная задача создания расширений для криптографии стояла и перед дизайнерами архитектуры RISC-V \cite{RISCV_AES}, что привело к созданию документа RISC-V Cryptography Extensions в 2021. Расширение включает в себя набор инструкций для ускорения алгоритмов AES, SM4 Block Cipher. Вычислений в полях: Carry-less multiply, Bitmanip instructions for Cryptography и пр \cite{RISCV_CRYPTO}. 

Расширяемая природа архитектуры RISC-V позволяет создавать и реализовывать собственные расширения системы команд.  И такие расширения активно создаются. Примером может служить криптографическое расширение LightWeight
Cryptography (LWC) от Национального института стандартов и технологий США (The National Institute of Standards and Technology, NIST) для использования в IoT (интернет вещей, internet of things) \cite{RISCV_IoT}.

Не смотря на существенное ускорение, вычисление на CPU векторов более 512 бит приводит к экспоненциальному росту сложности проектирования чипа. Этот факт является принципиальным ограничением криптографических систем на основе процессоров общего назначения. В этом смысле инициатива перешла к более узкоспециализированным вычислителям, вроде GPGPU.


\subsection{GPGPU}

GPGPU (General-purpose computing on graphics processing units) — использование графического процессора видеокарты, предназначенного для компьютерной графики, в целях производства математических вычислений, которые обычно проводит центральный процессор (CPU). Аппаратные решения, реализующие данную технологию, позволяют эффективно использовать параллелизм решаемых задач. Наиболее широко используются в сфере машинного обучения.

Список производителей данных вычислителей: NVIDIA, AMD, Cerebras, Google, Intel, Biren. Однако на текущий момент более $80\%$ мирового рынка производителей GPGPU вычислителей занято компанией NVIDIA. А ключевую роль в этом играет CUDA (Compute Unified Device Architecture) -- программная модель NVIDIA. И на сегодняшний день альтернатив

Классические криптографические алгоритмы, вроде RSA и AES, успешно оптимизируются под вычисления на существующих GPGPU решениях. CUDA реализации данных алгоритмов осуществляют на порядки превосходят передовые CPU решения.

Параллельная RSA-расшифровка данных с применением видеокарт NVIDIA показывает 1197.5x ускорение относительно CPU \cite{RSA_CUDA}. Скорость шифрования в AES достигает значений в $200 \text{Gbps}$ \cite{AES_CUDA} \cite{AES_Pascal}, что в десятки раз превосходит оптимизированные CPU реализации с использованием Intel AES Extension. 

Гетерогенные вычисления AES с применением связки CPU-GPU показывают снижение энергопотребления на $74\%$ в сравнении с CPU-only решением и на $21\%$ в сравнении с GPU-only системами \cite{GPGPU_EnergyEfficiency}.

\begin{figure}[h]
	\centering
	\includegraphics[width=0.7\linewidth]{images/quantum-computing-cupqc}
	\caption{Измерения на чипе NVIDIA H100. Сравнение с современными однопоточными процессорами, используемыми в тестовом пакете liboqs \cite{cuPQC}.}
	\label{fig:quantum-computing-cupqc}
\end{figure}
    
Недавнее сотрудничество компании QuSecure, лидера в сфере постквантовой криптографии, с компанией NVIDIA, привело к серьезным шагам в сторону адаптации GPGPU вычислителей для задач <<Постквантовой Эры>> \cite{cuPQC_Blog}. Одно из основных достижений в рамках сотрудничества -- создание CUDA cuPQC -- SDK (Software Development Kit) оптимизированных библиотек для ускорения передовых постквантовых алгоритмов. Заявляется $300-500x$ ускорение алгоритма Kyber 768 относительно оптимизированных с помощью Intel AVX реализаций (см. рис. \ref{fig:quantum-computing-cupqc}).  

\subsection{Квантовые решения}

На текущий момент существует два класса задач, которые успешно решаются существующими квантовыми решениями: генерация случайных чисел (ГСЧ) и квантовое распределение ключей.

\subsubsection{Генерация случайных чисел}

Важность генерации случайных чисел сложно переоценить. Случайные числа являются важнейшим ресурсом в огромном числе практических приложений. Последовательности случайных чисел применяются в системах безопасности, криптографии, в научных исследованиях (статистике, моделировании различных систем и процессов).

Генераторы случайных чисел традиционно делят на две категории: аппаратные (АГСЧ, англ. hardware random number generator, HRNG) и псевдослучайные (ПГСЧ, англ. pseudorandom number generator, PRNG).

Устройства, основанные на макроскопических случайных процессах, не могут обеспечить скорости получения случайных чисел, достаточной для прикладных задач. Поэтому в основе современных АГСЧ лежат источники шума, из которых извлекаются случайные биты: дробовой шум, радиоактивный распад, спонтанное параметрическое рассеяние \cite{Henk}.

Основная проблема аппаратных генераторов случайных чисел — это их относительно медленная по сравнению с генераторами псевдослучайных последовательностей работа. Также многие из них постепенно деградируют со временем, а анализ и верификация таких генераторов задача весьма затруднительная.

Квантовые генераторы случайных чисел (КГСЧ) в качестве физического источника энтропии используют квантовые процессы, которые сами по себе имеют вероятностную природу, что делает их идеально подходящими для криптографических приложений.

Решения Quantis QRNG Chip "ветерана" отрасли -- компании ID Quantique (IDQ), достигают скорости генерации $20 Mb/s$ \cite{Quantis_QRNG}. Не уступает им и ближайший конкурент Crypta Labs, предоставляя генераторы Firefly PCIe QRNG со скоростью генерации до $20 Mb/s$ \cite{CryptaLabs_Firefly}.

Если требуется высокая скорости генерации, то по этому критерию на порядок обходит конкурентов компания Quantum Dice со своими генераторами VERTEX 1100 \cite{QuantumDice_VERTEX} и APEX 2100 \cite{QuantumDice_APEX}, достигающими скорости генерации случайной последовательности бит $2.66 Gb/s$ и $7.5 Gb/s$ соответственно.

Активная работа по созданию передового КГСЧ сегодня ведется отечественной компанией QRate. Скорость их решения QRate Chaos будет превышать $1 Gb/s$ \cite{QRateChaos}.  


\subsubsection{Квантовое распределение ключей}

Квантовое распределение ключей (КРК) — криптографический протокол, позволяющий двум удаленным абонентам выработать общий случайный секрет (ключ).

Существует риск утраты конфиденциальности, связанный с компрометацией ключа, в том числе и неявной: увольнение сотрудника, который имел допуск к ключевой информации, необходимость допуска сотрудников других организаций и так далее. Чем дольше используется один и тот же ключ, тем выше вероятность того, что он уже явно или неявно скомпрометирован. Но помимо риска компрометации при длительном использовании одного и того же ключа существует ограничение на объем шифруемой информации, когда взлом невозможно осуществить статистическими методами.


\begin{figure}[h]
	\centering
	\includegraphics[width=0.7\linewidth]{images/infotecs_distrib}
	\caption{Квантовое распределение ключей. Принцип действия \cite{Infotecs_Sheet}.}
	\label{fig:infotecskeys}
\end{figure}

Секретность выработки квантовых ключей основана на следующих принципах:

Невозможно клонировать неизвестное квантовое состояние.

Невозможно различить два неортогональных квантовых состояния.

Невозможно измерить квантовое состояние без его изменения (редукция волновой функции).

Это делает невозможным атаку «человек по середине», другими словами незаметную компрометацию ключа шифрования.


Теорема о запрете клонирования неизвестного квантового состояния позволяет гарантированно детектировать пассивного/активного злоумышленника.





\subsection{Специализированное железо}
\section{Заключение}


\newpage
%\LaTeX{} \cite{latex2e} is a set of macros built atop \TeX{} \cite{texbook}.
\bibliographystyle{IEEEtran} % We choose the "plain" reference style
%\bibliography{IEEEabrv, refs} % Entries are in the refs.bib file
\bibliography{refs}


%https://www.cnbc.com/2024/06/02/nvidia-dominates-the-ai-chip-market-but-theres-rising-competition-.html
\end{document}



