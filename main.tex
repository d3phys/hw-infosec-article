\documentclass[a4paper,12pt]{article}


\usepackage[english,russian]{babel}	% локализация и переносы
%\usepackage{minipage}

\usepackage{unicode-math}
\usepackage{fontspec}
\setmainfont{Garamond}
\setmathfont{Garamond-Math}

%%% Гиперссылки
\usepackage{hyperref}
\usepackage[usenames,dvipsnames,svgnames,table,rgb]{xcolor}
\hypersetup{				% Гиперссылки
	unicode=true,           % русские буквы в разделареальных установках
	pdfproducer={Производитель}, % Производитель
	pdfkeywords={keyword1} {key2} {key3}, % Ключевые слова
	colorlinks=true,       	% false: ссылки в рамках; true: цветные ссылки
	linkcolor=red,          % внутренние ссылки
	citecolor=green,        % на библиографию
	filecolor=magenta,      % на файлы
	urlcolor=cyan           % на URL
}

\usepackage{geometry} % Простой способ задавать поля
\geometry{top=25mm}
\geometry{bottom=25mm}
\geometry{left=24mm}
\geometry{right=24mm}

\setcounter{secnumdepth}{2}

\usepackage{tikz}

\begin{document}
	
\begin{center}
	
        \normalsize Московский физико-технический институт \\
						(государственный университет)
	\vspace{46ex}
	

	\Large \textbf{Аппаратные решения для задач \\ защиты информации}

	\vspace{40ex}
\end{center}
\begin{flushright}
	\normalsize{Автор: Дедков Денис Андреевич \\ студент Б01-108 \\}
\end{flushright}

\vfill

\begin{center}
	Долгопрудный, 2024
\end{center}

\thispagestyle{empty} % выключаем отображение номера для этой страницы

\newpage

\tableofcontents{}
\newpage

\section{Введение}

Акцент поставлен на развивающихся и актуальных областях и на нерешенных задачах.

Обзор существующих аппаратных решений для актуальных задач в сфере защиты информации.


\section{Причины использования аппаратных решений}





% Definition of circles
\def\firstcircle{(0,0) circle (2.5cm)}
\def\secondcircle{(0:4cm) circle (2.5cm)}
\def\secondcircle{(0:4cm) circle (2.5cm)}

\colorlet{circle edge}{blue!50}
\colorlet{circle area}{blue!20}

\tikzset{filled/.style={fill=circle area, draw=circle edge, thick},
	outline/.style={draw=circle edge, thick},
	mycirc/.style={circle,fill=black!20, minimum size=0.01cm}}

\setlength{\parskip}{5mm}

\begin{center}
	\begin{tikzpicture}
		
		\node[label=above:{GPGPU}] (gpgpu) at (-1cm,3.6cm) {} ;
		\node[label=above:{FPGA}] (fpga) at (3cm,3.5cm) {} ;
		\node[label=above:{CPU Extensions}] (cpu) at (2cm,4.5cm) {} ;
		\node[label=above:{Квантовые решения}] (quantum) at (7cm,3cm) {};
		
		\draw (0,0) circle (3cm) node (accelerators) {Ускорители} ;
		\draw (4.5cm,0) circle (3cm) node (opportunitie) {Усилители} ;
		
		m	\draw (-1.2cm,1cm) circle (0.1cm) node (gpgpu_p) {} ;
		\draw (2.5cm,1cm) circle (0.1cm) node (fpga_p) {} ;
		\draw (0.5cm,1.5cm) circle (0.1cm) node (cpu_p) {} ;
		\draw (6cm,1.5cm) circle (0.1cm) node (quantum_p) {} ;
		
		\draw[->] (gpgpu) to [bend right=26] (gpgpu_p) ;	
		\draw[->] (fpga)  to [bend left=20] (fpga_p) ;	
		\draw[->] (cpu)  to [bend left=10] (cpu_p) ;	
		\draw[->] (quantum)  to [bend right=25] (quantum_p) ;	
	\end{tikzpicture}
\end{center}

\section{Обзор существующих аппаратных решений}

\subsection{Расширения CPU}

\subsection{GPGPU}

GPGPU (General-purpose computing on graphics processing units) — использование графического процессора видеокарты, предназначенного для компьютерной графики, в целях производства математических вычислений, которые обычно проводит центральный процессор (CPU). Аппаратные решения, реализующие данную технологию, позволяют эффективно использовать параллелизм решаемых задач. Наиболее широко используются в сфере машинного обучения.

Список производителей данных вычислителей: NVIDIA, AMD, Cerebras, Google, Intel, Biren. Однако на текущий момент более $80\%$ мирового рынка производителей GPGPU вычислителей занято компанией NVIDIA. А ключевую роль в этом играет CUDA (Compute Unified Device Architecture) -- программная модель NVIDIA. И на сегодняшний день альтернатив

Классические криптографические алгоритмы, вроде RSA и AES, успешно оптимизируются под вычисления на существующих GPGPU решениях. CUDA реализации данных алгоритмов осуществляют на порядки превосходят передовые CPU решения.

Параллельная RSA-расшифровка данных с применением видеокарт NVIDIA показывает 1197.5x ускорение относительно CPU \cite{RSA_CUDA}. Скорость шифрования в AES достигает значений в $200 \text{Gbps}$ \cite{AES_CUDA} \cite{AES_Pascal}, что в десятки раз превосходит оптимизированные CPU реализации с использованием Intel AES Extension. 
    
Недавнее сотрудничество компании QuSecure, лидера в сфере постквантовой криптографии, с компанией NVIDIA, привело к серьезным шагам в сторону адаптации GPGPU вычислителей для задач <<Постквантовой Эры>> \cite{cuPQC_Blog}. Одно из основных достижений в рамках сотрудничества -- создание CUDA cuPQC -- SDK (Software Development Kit) оптимизированных библиотек для ускорения передовых постквантовых алгоритмов . Завяляется $300-500x$ ускорение алгоритма Kyber 768 относительно оптимизированных с помощью Intel AVX реализаций \cite{cuPQC}.  


\subsection{Квантовые решения}

На текущий момент существует два класса задач, которые успешно решаются существующими квантовыми решениями: генерация случайных чисел (ГСЧ) и квантовое распределение ключей.

\subsubsection{Генерация случайных чисел}
\subsubsection{Квантовое распределение ключей}

\subsection{Специализированное железо}
\section{Заключение}


\newpage
%\LaTeX{} \cite{latex2e} is a set of macros built atop \TeX{} \cite{texbook}.
\bibliographystyle{plain} % We choose the "plain" reference style
\bibliography{refs} % Entries are in the refs.bib file

%https://www.cnbc.com/2024/06/02/nvidia-dominates-the-ai-chip-market-but-theres-rising-competition-.html
\end{document}



